\documentclass[a4paper, 11pt]{book}
\usepackage{comment} 
\usepackage{fullpage}
\usepackage[utf8x]{inputenc}
\usepackage[T2A]{fontenc}
\usepackage[english,russian]{babel}
\textwidth=15cm
\pagestyle{empty}
\topmargin=-1.5cm
\oddsidemargin=0pt
\parindent=0pt
\setcounter{chapter}{3}
\setcounter{section}{5}
\setcounter{subsection}{2}
\setcounter{equation}{35}
\usepackage[fleqn]{amsmath} 
\usepackage{color}
\definecolor{light-gray}{rgb}{0.8,0.8,0.8}

\begin{document}
\large\textit{3.5 Predicting the period of a pendulum}\hfill \textbf{49} \\
\vspace{15pt}

the pendulum’s amplitude $\theta_{0}$ is already a dimensionless group, so it can
affect the period of the system.\\


\colorbox{light-gray}{
\begin{minipage}{\textwidth}

\textbf{Problem 3.30~~~Choosing dimensionless groups}

Check that period T, length l, gravitational strength g, and amplitude $\theta_{0}$ produce
two independent dimensionless groups. In constructing useful groups for
analyzing the period, why should T appear in only one group? And why should
$\theta_{0}$ not appear in the same group as T?

\end{minipage} }
\vspace{5pt}

Two dimensionless groups produce the general dimensionless form
\begin{equation}
\textrm {one group = function of the other group,} 
\end{equation}
so
\begin{equation}
\frac{T}{\sqrt{l/g}}=
\textrm{function of $\theta_{0}$}.
\end{equation}
\vspace{5pt}

Because $T/{\sqrt{l/g}}=2\pi $ when $\theta_{0}=0$ (the small-amplitude limit), factor out
the  to simplify the subsequent equations, and define a dimensionless
period h as follows:
\begin{equation}
\frac{T}{\sqrt{l/g}}=2\pi h(\theta_{0}).
\end{equation}

The function $h$ contains all information about how amplitude affects the
period of a pendulum. Using h, the original question about the period becomes
the following: Is $h$ an increasing, constant, or decreasing function
of amplitude? This question is answered in the following section.

\subsection{Large amplitudes: Extreme cases again}

For guessing the general behavior of $h$ as a function of amplitude, useful
clues come from evaluating $h$ at two amplitudes. One easy amplitude is
the extreme of zero amplitude, where $h(0) = 1$. A second easy amplitude
is the opposite extreme of large amplitudes.\\


\it {How does the period behave at large amplitudes? As part of that question, what
is a large amplitude?}\\


\rm An interesting large amplitude is $\pi/2$, which means releasing the pendulum
from horizontal. However, at $\pi/2$ the exact $h$ is the following awful
expression (Problem 3.31):

\newpage
\large\textbf{50} \hfill \textit{3 Lumping} \\
\vspace{0pt}

\begin{equation}
h\left ( \pi /2 \right )= \frac{\sqrt{2}}{\pi }\int_{0}^{\pi /2}\frac{\mathrm{d} \theta }{\sqrt{cos\theta }}.
\end{equation}

Is this integral less than, equal to, or more than 1? Who knows? The integral
is likely to have no closed form and to require numerical evaluation
(Problem 3.32).

\vspace{10pt}

\colorbox{light-gray}{
\begin{minipage}{\textwidth}

\textbf{Problem 3.31~~~General expression for $h$}

Use conservation of energy to show that the period is
\begin{equation}
T(\theta _{0})=2\sqrt{2}\sqrt{\frac{l}{g}}\int_{0}^{\theta _{0}}\frac{\mathrm{d} \theta _{0}}{\sqrt{cos\theta -cos\theta _{0}}}.
\end{equation}
Confirm that the equivalent dimensionless statement is
\begin{equation}
h(\theta _{0})=2\sqrt{2}\sqrt{\frac{l}{g}}\int_{0}^{\theta _{0}}\frac{\mathrm{d} \theta _{0}}{\sqrt{cos\theta -cos\theta _{0}}}.
\end{equation}
For horizontal release, $\theta _{0}=\pi/2$, and
\begin{equation}
h\left ( \pi /2 \right )= \frac{\sqrt{2}}{\pi }\int_{0}^{\pi /2}\frac{\mathrm{d} \theta }{\sqrt{cos\theta }}.
\end{equation}
\textbf{Problem 3.32~~~Numerical evaluation for horizontal release}

Why do the lumping recipes (Section 3.2) fail for the integrals in Problem 3.31?
Compute $h(\pi/2)$ using numerical integration.

\end{minipage} }
\vspace{5pt}

Because $ \theta _{0}= \pi/2$ is not a helpful extreme, be even more extreme. Try $ \theta _{0}= \pi$, which means releasing the pendulum bob from vertical. If the
bob is connected to the pivot point by a string, however, a vertical release
would mean that the bob falls straight down instead of oscillating. This
novel behavior is neither included in nor described by the pendulum
differential equation.
\vspace{6pt}

Fortunately, a thought experiment is cheap to improve:
Replace the string with a massless steel
rod. Balanced perfectly at $ \theta _{0}= \pi$, the pendulum
bob hangs upside down forever, so $T(\pi) = ∞$ and
$h(\pi) = ∞$. Thus, $h(\pi) > 1$ and $h(0) = 1$. From
these data, the most likely conjecture is that h increases
monotonically with amplitude. Although
h could first decrease and then increase, such twists and turns would
be surprising behavior from such a clean differential equation. (For the
behavior of h near $ \theta _{0}= \pi$, see Problem 3.34).

\newpage
\large\textit{3.5 Predicting the period of a pendulum}\hfill \textbf{51} \\
\vspace{15pt}

\colorbox{light-gray}{
\begin{minipage}{\textwidth}

\textbf{Problem 3.33~~~Small but nonzero amplitude}


As the amplitude approaches $\pi$, the dimensionless period $h$
diverges to infinity; at zero amplitude, $h = 1$. But what about
the derivative of $h$? At zero amplitude $(\theta_{0} = 0)$, does $h(\theta_{0})$
have zero slope (curve $A$) or positive slope (curve $B$)?
\vspace{20pt}

\textbf{Problem 3.34~~~Nearly vertical release}

Imagine releasing the pendulum from almost vertical:
an initial angle $\pi$ − $\beta$ with $\beta$ tiny. As a function of $\beta$,
roughly how long does the pendulum take to rotate by
a significant angle—say, by 1 rad? Use that information
to predict how $h(\theta_{0})$ behaves when $\theta_{0}
≈ \pi$. Check and
refine your conjectures using the tabulated values. Then
predict $h(\pi−10^{−5})$.
\end{minipage} }

\subsection{Moderate amplitudes: Applying lumping}

The conjecture that h increases monotonically was derived using the extremes
of zero and vertical amplitude, so it should apply at intermediate
amplitudes. Before taking that statement on faith, recall a proverb from
arms-control negotiations: “Trust, but verify.”\\

\it{At moderate (small but nonzero) amplitudes, does the period, or its dimensionless
cousin h, increase with amplitude?}\\

\rm In the zero-amplitude extreme, $sin~\theta$ is close to $\theta$. That approximation
turned the nonlinear pendulum equation
\begin{equation}
\frac{\mathrm{d}^2\theta }{\mathrm{d} t^2}+\frac{g}{l}~sin~\theta=0
\end{equation}

into the linear, ideal-spring equation—in which the period is independent
of amplitude.
\vspace{6pt}

At nonzero amplitude, however, $\theta$ and $sin \theta$ differ and their difference
affects the period. To account for the difference and predict the period,
split $sin \theta$ into the tractable factor $\theta$ and an adjustment factor $f(\theta)$. The resulting equation is
\begin{equation}
\frac{\mathrm{d}^2\theta }{\mathrm{d} t^2}+\frac{g}{l}\theta\underbrace{\frac{sin \theta}{\theta}}_{f(\theta)}=0.
\end{equation}

\newpage
\large\textbf{52} \hfill \textit{3 Lumping} \\
\vspace{15pt}

The nonconstant $f(\theta)$ encapsulates the nonlinearity of $f(\theta)$
the pendulum equation. When $\theta$ is tiny, $f(\theta) ≈ 1$: The
pendulum behaves like a linear, ideal-spring system.
But when $\theta$ is large, $f(\theta)$ falls significantly below 1,
making the ideal-spring approximation significantly
inaccurate. As is often the case, a changing process is
difficult to analyze—for example, see the awful integrals in Problem 3.31.
As a countermeasure, make a lumping approximation by replacing the
changing $f(\theta)$ with a constant.\\

The simplest constant is$f(\theta)$. Then the pendulum differential equation becomes
\begin{equation}
\frac{\mathrm{d}^2\theta }{\mathrm{d} t^2}+\frac{g}{l}\theta=0.
\end{equation}\\

This equation is, again, the ideal-spring equation. In this approximation, period does not depend on amplitude, so $h = 1$ for
all amplitudes. For determining how the period of an unapproximated
pendulum depends on amplitude, the $f(\theta) → f(0)$ lumping approximation
discards too much information.\\

Therefore, replace $f(\theta)$ with the other extreme
$f(\theta_0)$. Then the pendulum equation becomes
\begin{equation}
\frac{\mathrm{d}^2\theta }{\mathrm{d} t^2}+\frac{g}{l}\theta f(\theta_0)=0.
\end{equation}\\
\it {Is this equation linear? What physical system does
it describe?}\\

\rm Because $f(\theta_0)$ is a constant, this equation is linear! It describes a zeroamplitude
pendulum on a planet with gravity geff that is slightly weaker
than earth gravity—as shown by the following slight regrouping:
\begin{equation}
\frac{\mathrm{d}^2\theta }{\mathrm{d} t^2}+\overbrace{\frac{g\mathrm{f}(\theta_0)}{\theta}}^{g_{e\mathrm{ff}}}\theta=0.
\end{equation}\\
Because the zero-amplitude pendulum has period $T=2\pi \sqrt{l/g}$, the zeroamplitude,
low-gravity pendulum has period
\begin{equation}
T(\theta_0)\approx 2\pi \sqrt{\frac{l}{g_{e\mathrm{ff}}}}=2\pi \sqrt{\frac{l}{g\mathrm{f(\theta_0)}}}.
\end{equation}



\end{document}
